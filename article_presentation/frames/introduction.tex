%---------------------------------- Large Language Models -------------------------------
\begin{frame}{Large Language Models}
\begin{columns}
      
    \begin{column}[t]{0.4\textwidth}
    \begin{block}{Summary}
    
        \begin{itemize}
            \item State-of-the-art of Natural Language Processing (NLP) problems
            \item Architecture : Transformers\footnote{Vaswani et al., « Attention is All you Need ».} block, mixed with classical layers (MLP, Conv)
            \item Huge size : Billions of parameters (1B to 405B for Llama 3)
            \item 2 phases of training : pre-training and \textbf{fine-tuning}
        \end{itemize}
            

    \end{block}
    \end{column}
        
    \begin{column}[t]{0.55\textwidth}
    \begin{block}{Self Attention }

        \begin{figure}
            \centering
            \input{imgs/self_attention.tex}
            \caption{Self Attention mecanism illustration}
        \end{figure}
    
        Self attention is the key of LLM, used to compute the context of each token.
    \end{block}  
    \end{column}
         
\end{columns}
\end{frame}

%---------------------------------- Fine-tuning Workflow -------------------------------

\begin{frame}{Fine-Tuning}

    Following a first phase of pre-training, Fine-tuning is used to correct behavior or add in-domain data to a model, with limited resources. 


    \begin{figure}
        \centering
        \resizebox{\textwidth}{!}{
            \input{imgs/pre_training}
        }
        \caption{Pre-training and Fine-tuning generic workflow}
    \end{figure}  
        

    
\end{frame}

%---------------------------------- Fine Tuning Frame -------------------------------
\begin{frame}{Parameters Efficient Fine-Tuning (PEFT)}
    Set of methods aims to reduce the computation cost of fine-tuning. 2 main approaches : \textit{Additive} and \textbf{reparametrization}.
    
    \begin{columns}  
  
        \begin{column}[t]{0.45\textwidth}
        \begin{block}{Reparametrization}
            Use lower-cost proxy as trainable weights, and merge at the end. e.g. : LoRA and derived methods
        \end{block}
        \end{column}
    
        \begin{column}[t]{0.45\textwidth}
        \begin{block}{Additive}
            Add part of the model, often linear layer, to train these.  One con is to add inference to generation.
            
        \end{block}
        \end{column}
      
    \end{columns}

    \begin{block}{Quantization}
        To reduce further the cost of computing during the training, quantization can also be used. This can be combined with either of precedent approaches. 
        
    \end{block}

\end{frame}



%---------------------------------- LoRA -------------------------------
\begin{frame}{Low Rank Adaptation (LoRA)}
    \begin{block}{Principle}
        Merging Fine-tuning layers with pre-trained ones can be written as $W = W_0 + \Delta W$, with $W_0$ the pre-trained weights and $\Delta W$ the fine-tuned ones. With LoRA, $W=W_0 + \frac{\alpha}{r} B.A$        
    \end{block}

    \begin{columns}
        \begin{column}[t]{0.45\textwidth}
        \begin{figure}
            \centering
            \resizebox{\textwidth}{!}{
                \input{imgs/lora}
            }
            \caption{LoRA Decomposition}
        \end{figure}
            
        \end{column}
        
        \begin{column}[t]{0.3\textwidth}
            \begin{block}{LoRA hyperparameters}
            \begin{itemize}
                \item rank $r$ : the common dimension between $A$ and $B$.
                \item alpha $\alpha$ : apply a weighting between fine-tuning and pre-trained weights
            \end{itemize}
                
            \end{block}
            
        \end{column}
    \end{columns}
    
\end{frame}

%---------------------------------- HPO -------------------------------
\begin{frame}{Hyperparameter Optimization (HPO)}

    \begin{columns}
         
        %%%%%%%%%%%%%%%%%%%%%%%%%% COLONNE DE GAUCHE %%%%%%%%%%%%%%
           \begin{column}{0.3\textwidth} 
           \begin{block}{Objectives}
            \begin{itemize}
                \item Better performance than manual tuning
                \item Ease popularization of the Fine Tuning
            \end{itemize}
            
           \end{block}
    
           \end{column}
               
        %%%%%%%%%%%%%%%%%%%%%%%%% COLONNE DE DROITE %%%%%%%%%%%%%%
           \begin{column}{0.7\textwidth}
            \begin{figure}
                \centering
                \resizebox{\textwidth}{!}{
                    \newcommand{\Dtrain}{\mathcal{D}_{train}}
\newcommand{\Dval}{\mathcal{D}_{val}}
\newcommand{\model}{\mathcal{M}}

\begin{tikzpicture}
    \tikzstyle{data}=[rectangle split, rectangle split parts = 2,draw,text centered, fill=yellow!20]
    \tikzstyle{model} = [rectangle, draw, text centered, fill = blue!20]
    \tikzstyle{function} = [rectangle, draw, text centered, fill = red!20, font = \bfseries]
    \tikzstyle{metrics} = [rectangle, text centered, draw, fill=teal!20]
    
\tikzstyle{dot_arrow} = [thick,dotted,->,>=stealth]

\node (train_data) [data]
    {
        \textbf{Training Data}
        \nodepart{second} Alpaca};  

\node (PT_model)[model, below of = train_data]{Pre Trained Model };

\node (training) [function, right of = PT_model, anchor = west, xshift = 1.7cm]{Training};

\node (hp) [metrics, above of = training]{hyperparameters};

\node (FT_model) [model, right of = training, anchor = west, xshift = 0.7cm]{Fine Tuned model};

\node (val_data) [data, above of = FT_model, xshift = 0.5cm]
    {
        \textbf{Validation Data}
        \nodepart{second} Hellaswag}; 
        
\node (evaluate) [function, right of = FT_model, anchor = west, xshift = 1cm]{Evaluate};

\node (metrics) [metrics, right of = evaluate, anchor = west, xshift = 0.3cm]{metrics};


\begin{scope}[on background layer]
    \node(bbfunction)[draw, thick,fill=black!10,draw=black!20, dashed, rounded corners, fit=(train_data) (PT_model) (training) (FT_model)(val_data)(evaluate)(metrics), inner sep=0.3cm, label=below:{Black-Box function  }] {};
\end{scope}

\draw [dot_arrow] (train_data) -- (training);
\draw [dot_arrow] (PT_model) -- (training);
\draw [dot_arrow] (hp) -- (training);
\draw [dot_arrow] ([xshift = -1.9cm]bbfunction.north) -- (hp.north);
\draw [dot_arrow] (training) -- (FT_model);
\draw [dot_arrow] (val_data) -- (evaluate);
\draw [dot_arrow] (FT_model) -- (evaluate);
\draw [dot_arrow] (evaluate) -- (metrics);
\draw [dot_arrow] (metrics) -- ([xshift = 6.5cm]bbfunction.north);

\node (hpo) [circle, above of = bbfunction, yshift = 2cm,xshift=2.2cm, draw, fill = teal!40]{HPO};
\node [left of = hpo, yshift = 0.25cm, anchor = east]{solution};
\node [right of = hpo, yshift = 0.25cm,anchor = west]{validation accuracy};

%\draw [thick,->,>=stealth]    ([xshift = 6.25cm]bbfunction.north) to[out=90,in=-5] (hpo.east);
\draw [thick,<-,>=stealth]     (hpo.east)to[out=0,in=90] ([xshift = 6.5cm]bbfunction.north);
\draw [thick,->,>=stealth]     (hpo.west) to[out=180,in=90] ([xshift = -1.9cm]bbfunction.north);




\end{tikzpicture}
                }
                \caption{HPO workflow}
           \end{figure}  
           \end{column}
                
       \end{columns}

\end{frame}

%---------------------------------- Review Summary -------------------------------
\begin{frame}{Related Works}
    \begin{figure}
        \resizebox{0.9\textwidth}{!}{
            \begin{tikzpicture}[node distance = 1.15cm]

\tikzstyle{field} = [rectangle,rounded corners, minimum width=2cm, minimum height=0.8cm, text centered, draw=black]
\tikzstyle{art} = [minimum width=2cm, minimum height=0.8cm, text centered, draw=black, fill=blue!30]
\tikzstyle{arrow} = [thick, ->, >=stealth]


% base
\node (base)[field, fill = violet!30]{LLM and Optimization};

% lvl 1
\node (opt_llm)[field, below of = base, xshift = -3cm, fill = blue!30]{Optimization to LLM};

\node (llm_opt)[field, below of = base, xshift = 3cm, fill = purple!30]{LLM to Optimization};

% lvl 2
\node (autodnn)[field, below of = opt_llm, xshift = -0.5cm, fill=teal!30]{AutoDNN};
\node(prompt)[art,right of = autodnn, xshift = 2.5cm, fill = blue!20]{
    \textbf{Prompt Eng.}
};
\node (gen_ea)[art, below of = llm_opt, fill = purple!20]{
    \textbf{Generate EA}
};

% lvl 3
\node (hpo)[field, below of = autodnn, xshift = -1.5cm, fill = cyan!50]{HPO};
\node (nas)[field, right of = hpo, xshift = 3.5cm, fill = green!30]{NAS};

% lvl4 hpo

\node (hpo_ft)[art, below of = hpo, fill = cyan!35]{
    \textbf{Fine-Tuning}
};
\node (hpo_pt)[art, left of = hpo_ft, xshift = -1cm, fill = cyan!20]{
    \textbf{Pre-training}
};
\node (hpo_gen)[art, right of = hpo_ft,xshift = 1cm, fill = cyan!20]{
    \textbf{Generation}
};
% lvl4 nas
\node (nas_scratch)[art, below of = nas, fill = green!20]{
    \textbf{From Scratch}
};
\node (nas_pruning)[art, right of = nas_scratch,xshift = 1cm, fill = green!20]{
    \textbf{Pruning}
};

% arrows
\draw[arrow] (base) -- (opt_llm);
\draw[arrow] (base) -- (llm_opt);

\draw[arrow] (opt_llm) -- (autodnn);
\draw[arrow] (opt_llm) -- (prompt);
\draw[arrow] (llm_opt) -- (gen_ea);

\draw[arrow] (autodnn) -- (hpo);
\draw[arrow] (autodnn) -- (nas);

\draw[arrow] (hpo) -- (hpo_pt);
\draw[arrow] (hpo) -- (hpo_ft);
\draw[arrow] (hpo) -- (hpo_gen);

\draw[arrow] (nas) -- (nas_scratch);
\draw[arrow] (nas) -- (nas_pruning);

% fitting box
\node (current)[draw,thick, dashed, rounded corners, 
    fit = (hpo)(hpo_ft), inner sep = 0.2cm,
    label = below :{Current work}]{};

\end{tikzpicture}
        }
        \caption{Summary of links between LLM and Optimization}
    \end{figure}
    
    
\end{frame}