\chapter{Company's presentation}
\label{chap:inria}

\epigraph{What is research but a blind date with knowledge?}{Will Harvey}
INRIA, National Research Institute for Computer Science and Automation,is one of the leading public institutions involved in academic research in France. Today, more than 3,800 scientists, working in 220 project teams, are involved in digital research at INRIA. Since its creation, INRIA's mission has been to ensure French sovereignty and autonomy in IT-related fields, while transferring knowledge to the industrial world. 

INRIA is made up of 10 research centers, spread across France, which
work in a dozen areas, including :
\begin{multicols}{2}
\begin{itemize}
    \item High-performance computing
    \item Digital Education
    \item Artificial intelligence
    \item Digital health
    \item Data science
    \item Software
\end{itemize}
\end{multicols}

To support this, the institute is developing a large number of partnerships, playing its role as a research vector. These partnerships include, first and foremost university partnerships; the research centers are attached to universities, in order to contribute in the training of tomorrow's scientists. We can also add institutional research partners, such as CNRS or CEA in France, and many others in France, and many others in Europe, which enable us to take on large-scale projects. projects. And last but not least, our industrial partnerships help to keep the Institute going. These range from the 170 start-ups incubated on INRIA premises over the past 20 years, to industry giants such as Microsoft, for example, who collaborate in the joint Microsoft-Research / INRIA joint research center\cite{inria_microsoft}

\section{INRIA history}

INRIA was founded in 1967, under the name \say{Institut de Recherche en Informatique et Automatique} (IRIA), as part of the \textit{Plan Calcul}\cite{PlanCalcul}. This project, launched by the French government in 1966, was designed to ensure France's autonomy and sovereignty in the field of information technology.
A few years later, in the 70s, INRIA led the \textit{Cyclades} project, participating in the networking of computers, and contributing to what would later become the Transmission Control Protocol (TCP).

In 1979, the institute affirmed its commitment to a national structure, and became INRIA, with the opening of centers in Rennes, then Sophia-Antipolis, Nancy and Grenoble. The same dynamic led to the creation of Simulog, the first start-up incubated at INRIA, reaffirming the Institute's commitment to innovation in the broadest sense of the term.

In the early 2000s, INRIA, like the rest of the world, was strongly influenced by the development of the Web and its applications. In particular, INRIA was responsible for the European node of the W3C (World Wide Web Consortium). Since then, INRIA has diversified its research, supporting research into digital health and developing significant expertise in software engineering. 

\section{INRIA center of Lille University}
The INRIA center at the University of Lille was born of a partnership between INRIA and the University of Lille, in 2007. It first set up a site in Villeneuve d'Asq \footnote{40, avenue Halley - 59650 Villeneuve d'Ascq}, close to the scientific city campus, which also houses Polytech Lille and Centrale Lille, then a second site in Lille \footnote{172, avenue de Bretagne - 59000 Lille}, in the Euratechnologies park. It currently houses 385 employees, including 260 researchers and 50 engineers, divided into 15 project teams.
In line with national priorities, while at the same time asserting its specialties, the center focuses on 5 themes: 
\begin{multicols}{2}
    
\begin{itemize}
    \item Data Science
    \item Software Engineering
    \item Cyber Systems Physics
    \item Digital Health
    \item Digital Sobriety
\end{itemize}
\end{multicols}

\section{BONUS team}

The BONUS team (Big Optimization aNd Ultra Scale computing) is part of the INRIA center at the University of Lille, located on the Lille site, and comprises around 15 people, including 4 researchers. It focuses on large-scale optimization problems, which are characterized by: 1) a large number of problem dimensions 2) the possibility of multi-objective optimization 3) very high solution evaluation costs 4) the need for supercomputers.
To address these issues, the team has divided its research into 3 areas:
\begin{itemize}
  \item Decomposition-based optimization: defining and solving sub-problems to approximate the larger problem
    \item Optimization supported by machine learning: machine learning, and by extension artificial intelligence, concentrates a set of problems that require optimization.
    \item Large-scale computation: for these problems, the need for computation is reaching levels that seemed unimaginable just a few years ago. Parallelization and the development of suitable hardware are the key to progress.
\end{itemize}

INRIA, through its BONUS team, is involved in the PEPR NumPex\cite{numpex}, which aims to enable the development and use of exascale computing\footnote{machines capable of performing at least $10^{18}$ \acrshort{flops}}. This internship is part of the Exa-MA priority project, which aims to develop mathematical methods and algorithms for translating simulated phenomena into equations.
