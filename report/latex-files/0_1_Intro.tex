\chapter*{Introduction}
\markleft{Introduction}
\addcontentsline{toc}{chapter}{Introduction}

From September 09, 2024, to February 21, 2025, I joined the BONUS project-team, Big Optimization aNd Ultra-Scale computing, at the INRIA Center of the University of Lille, under the supervision of Mr. El-Ghazali Talbi. Adding a Master's degree in Systems Optimization and Safety to my Industrial Engineering curriculum, the intrinsic question was : do I want to continue in the research world ? From the point of view of building my career path, this internship is designed to answer this question.

%Institution historique de la recherche en France, l'INRIA se place en tant que figure de proue de la recherche liée à l'information et au numérique. C'est dans le cadre du Programme et Equipement Prioritaire de Recherche (PEPR) se désignant Numérique pour l'Exascale (NumPEx), et en particulier l'axe Exa-MA, que j'ai effectué ce stage. Sans rentrer trop tôt dans le détail, le but de mon stage est la poursuite de recherche d'architecture de réseaux de neurones, appliqué à des grands modèles de données (NAS/LLM).

As one of France's historic research institutions, INRIA is a leading figure in information and digital research. It was within the framework of the Priority Research Program and Equipment (PEPR) called Numérique pour l'Exascale (NumPEx), and in particular the Exa-MA axis, that I carried out this internship. Without going into too much detail, the aim of my internship is to pursue research into Neural Architectural Search (NAS), applied to Large Language Models (LLM).


Après une première phase présentant en détail l'INRIA, nous pourrons poursuivre par un focus sur le sujet du stage, et le contexte dans lequel il s'inscrit. 


