\chapter*{Introduction}
\markleft{Introduction}
\addcontentsline{toc}{chapter}{Introduction}

From September 09, 2024, to February 21, 2025, I joined the \acrfull{bonus} team at the INRIA Center of the University of Lille, under the supervision of Mr. El-Ghazali Talbi. Adding a Master's degree in Systems Optimization and Safety to my Industrial Engineering curriculum, the intrinsic question was : do I want to continue in this academic track ? From a carreer path point of view, this internship was designed to answer this question.

As one of the greatest french institute for research in digital science and technology, INRIA is a leading figure for such an internship. It was within the framework of the \acrfull{pepr} called \textit{Numérique pour l'Exascale (NumPEx)}, and in particular the Exa-MA axis, that I carried out this internship. Introduced by the \acrfull{nas} problem, the focus of my internship was to apply optimization algorithms to a recent paradigm in \acrfull{ai} and industrial world : \acrfull{llm}.

After a first phase presenting the INRIA, we will continue with a focus on the subject of the internship, and the context in which it is embedded, to fully understand the stakes and the environment. Then, a second phase will adress the methodology of the internship, to give an insight into the research process and the skills needed to navigate it effectively. This phase allow the final chapter to handles the results, and the prospectives approach of the internship.

