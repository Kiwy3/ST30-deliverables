%%%%%%%%%%%%%%%%%%%%%%%%%%%%%%%%% Conclusion  %%%%%%%%%%%%%%%%%%
\begin{frame}{Résultats du stage}
    \begin{columns}
        \begin{column}{0.45\textwidth}
            \begin{block}{Implémentation d'OHP pour du Fine Tuning de LLM}
                Preuve du concept d'utilisation des algorithmes utilisés, ainsi qu'une base pour de futur travaux de l'équipe
                
            \end{block}
        \end{column}
        \begin{column}{0.45\textwidth}
            \begin{block}{Comparaison entre approche bayésien/décomposition pour fonction couteuse}
                Incluant l'exploration d'une première partie de l'hybridation des deux

                Tend vers une généralisation d'une approche par décomposition améliorée par l'utilisation d'un \textit{surrogate}
            \end{block}
            
        \end{column}
    \end{columns}
\end{frame}

%%%%%%%%%%%%%%%%%%%%%%%%%%%%%%%%% Conclusion  %%%%%%%%%%%%%%%%%%
\begin{frame}{Apprentissage}
    \begin{columns}
        \begin{column}{0.45\textwidth}
            \begin{block}{1ere expérience de recherche}
                \begin{itemize}
                    \item Apprentissage de la rigueur
                    \item gestion de la littérature
                    \item première écriture ...
                \end{itemize}
                
            \end{block}
        \end{column}
        \begin{column}{0.45\textwidth}
            \begin{block}{Programmation pour une démarche de recherche}

                \begin{itemize}
                    \item Habitude de programmation pour l'optimisation globale
                    \item Prototypage et versionnage
                    \item Transmission pour l'équipe
                    \item Approche du paralellisme
                \end{itemize}
            \end{block}       
        \end{column}
    \end{columns}
\end{frame}

%%%%%%%%%%%%%%%%%%%%%%%%%%%%%%%%% Conclusion  %%%%%%%%%%%%%%%%%%
\begin{frame}{Poursuite du projet profressionel}
    \begin{columns}
        \begin{column}{0.45\textwidth}
            \begin{block}{Poursuite en recherche}
                Confirmation de l'attrait pour le domaine
                
            \end{block}
        \end{column}
        \begin{column}{0.45\textwidth}
            \begin{block}{Début d'une thèse en mars}
                    Sujet : Ecological and economic logistics service network design : Models and Decision Support Algorithms

                    Equipe INOCS, au sein de l'INRIA Lille, dirigé par Frederic Semet
                
            \end{block}   
        \end{column}
    \end{columns}
\end{frame}