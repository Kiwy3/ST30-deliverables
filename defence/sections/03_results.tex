%%%%%%%%%%%%%%%%%%%%%%%%%%%%%%%%% LHS results %%%%%%%%%%%%%%%%%%
\begin{frame}{Echantillonnage avec LHS}
    \begin{columns}
        %%%%%%%%%%%%%%%%%%%%%%%%%% COLONNE DE GAUCHE %%%%%%%%%%%%%%
        \begin{column}{0.4\textwidth} 
            \begin{block}{Principe}
                Explorer l'espace et proposer une borne inférieure
                
            \end{block}
            \begin{figure}
                \centering
                \input{assets/tikz_picture/lhs.tex}
                \caption{Illustration du Latin Hypercube Sampling avec $g=5$}
            \end{figure} 
     
            \end{column}
                 
         %%%%%%%%%%%%%%%%%%%%%%%%% COLONNE DE DROITE %%%%%%%%%%%%%%
            \begin{column}{0.5\textwidth}
                \begin{figure}
                    \centering
                    \includegraphics[width = \textwidth]{assets/imgs/plots/sampling/lhs.png}
                    \caption{Résumé des résultats par sampling}
                \end{figure} 
            \end{column}
                 
    \end{columns}
\end{frame}

%%%%%%%%%%%%%%%%%%%%%%%%%%%%%%%%% LHS results %%%%%%%%%%%%%%%%%%
\begin{frame}{Résultats des 3 algorithms}
    \begin{figure}
        \centering
        \includegraphics[width = 0.9\textwidth]{assets/imgs/plots/global/comparison.png}
        \caption{Résumé des résultats par sampling. {\footnotesize \textit Détails par algorithme dans les annexes \ref{ap:bo_results}, \ref{ap:soo_results} et \ref{ap:bamsoo_results}}}
    \end{figure} 
    .
\end{frame}

%%%%%%%%%%%%%%%%%%%%%%%%%%%%%%%%% Analyse %%%%%%%%%%%%%%%%%%
\begin{frame}{Analyse}
    \begin{table}[h!]
        \centering
        \begin{tabular}{|c||c|c||c|c|c|}
        \hline
           Jeu de données  & Borne Inf.$^1$& Borne Sup.$^2$ & BO-GP & SOO & BaMSOO \\
        \hline
           Hellaswag (validation)  & 47.90 & \textit{41.5} & 47.91 & 47.84 & 47.84\\
           MMLU (testing) & 37.61 & 49.3 & 38.11 & 37.42 & 37.50 \\
        \hline
        \end{tabular}
        \caption{Bornes et meilleurs résultats sur les 2 jeu de données}
    \end{table}
    {\footnotesize 1 : expérience avec LHS; 2 : Fine tuning par Meta}

    \begin{block}{Points clés}     
    \end{block}
    
    \vspace*{-10pt}
    \begin{columns}
        
        %%%%%%%%%%%%%%%%%%%%%%%%%% COLONNE DE GAUCHE %%%%%%%%%%%%%%
        \begin{column}{0.45\textwidth} 
                \begin{itemize}
                    \item Borne Sup. sur Hellaswag non pertinente
                    \item Seul BO arrive au dessus de LHS
                    \item BaMSOO n'améliore que peu SOO
                \end{itemize}
        \end{column}  
         %%%%%%%%%%%%%%%%%%%%%%%%% COLONNE DE DROITE %%%%%%%%%%%%%%
            \begin{column}{0.45\textwidth}
                \begin{itemize}
                    \item principe de BaMSOO fonctionnel (visible annexe \ref{ap:bamsoo_results})
                    \item Espace de solution n'évolue que peu, le retravailler pour mesurer pleinement la performance des algorithmes
                \end{itemize}
            \end{column}
        
                 
    \end{columns}
    
\end{frame}

%%%%%%%%%%%%%%%%%%%%%%%%%%%%%%%%% Perspectives %%%%%%%%%%%%%%%%%%

\begin{frame}{Perspectives}
    \begin{columns}
        
        %%%%%%%%%%%%%%%%%%%%%%%%%% COLONNE DE GAUCHE %%%%%%%%%%%%%%
        \begin{column}[t]{0.45\textwidth} 
            \begin{block}{Poursuite du travail}
                \begin{itemize}
                    \item Retour sur l'article et présentation en conférence (si validation)
                    \item Elargissement de l'espace de recherche
                    \item Diversification sur les modèles/données
                \end{itemize}                
            \end{block}

        \end{column}  
         %%%%%%%%%%%%%%%%%%%%%%%%% COLONNE DE DROITE %%%%%%%%%%%%%%
            \begin{column}[t]{0.45\textwidth}
                \begin{block}{Généralisation hors LLM}
                    \begin{itemize}
                        \item Parallelisation d'algorithme d'optimisation de fonction couteuse pour Exascale (project Exa-MA)
                        \item Intégration de substitut dans les algorithme parallèle à Partition \footnote[3]{\textit Partition-based Parallel Bayesian Optimisation}
                    \end{itemize}                
                \end{block}
            \end{column}
        
                 
    \end{columns}
    
\end{frame}